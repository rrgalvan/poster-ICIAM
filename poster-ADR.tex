%%%%%%%%%%%%%%%%%%%%%%%%%%%%%%%%%%%%%%%%%
% MUW Poster
% LaTeX Template
% Version 1.0 (31/08/2016)
% (Based on Version 1.0 (31/08/2015) of the Jacobs Portrait Poster
%
% License:
% CC BY-NC-SA 3.0 (http://creativecommons.org/licenses/by-nc-sa/3.0/)
%
% Created by:
% Nicolas Ballarini, CeMSIIS, Medical University of Vienna
% nicoballarini@gmail.com
% http://statistics.msi.meduniwien.ac.at/
%%%%%%%%%%%%%%%%%%%%%%%%%%%%%%%%%%%%%%%%%


\def\footer#1{\def\insertfooter{#1}}
% --------------------------------------------------------------------------------------
%	PACKAGES AND OTHER DOCUMENT CONFIGURATIONS
% --------------------------------------------------------------------------------------

\documentclass[final]{beamer}

\usepackage[scale=1.1]{beamerposter} % Use the beamerposter package


% Include a logo of your project if desired
% \logo{\pgfputat{\pgfxy(-6,93)}{\pgfbox[center,base]{\includegraphics[width=0.15\textwidth]{ProjectLogo.jpg}}}}

\usepackage{multicol}
\usepackage[utf8]{inputenc}
\usepackage{array}
% The following two are column definitions for the aknowledgements section
\newcolumntype{L}{>{\arraybackslash}m{22cm}}
\newcolumntype{S}{>{\arraybackslash}m{5cm}}
\usepackage{pgf}
\usepackage{mathtools}
\usepackage{amsmath, amsthm, amssymb, amsfonts, empheq}
\usepackage{exscale}
\usepackage{xcolor}
\usepackage{ushort}
\usepackage{setspace}
\usepackage[square,numbers]{natbib}
\usepackage{url}
\usepackage{multirow}
\usepackage{arydshln}
\usepackage{ragged2e} % Justify text
\addtobeamertemplate{block begin}{}{\justifying} % Justify all blocks
\renewcommand{\indent}{\hspace*{2em}}

\bibliographystyle{nat}
\renewcommand{\vec}[1]{\ushort{#1}}
\renewcommand{\vec}[1]{\mathbf{#1}}
\definecolor{greenMUW}{RGB}{60,191,174}
\definecolor{blueMUW}{RGB}{17,79,29}
\definecolor{skinMUW}{RGB}{234,208,207}
\definecolor{micolor}{RGB}{04,88,137}
\definecolor{hellblauMUW}{RGB}{225,133,33}
\setbeamercolor{block title}{bg=white, fg = hellblauMUW}

\setbeamercolor{alerted text}{fg=greenMUW!200}

% -----------------------------------------------
% START Set the colors
% Uncomment to apply colors you want to use.
% -----------------------------------------------
\colorlet{themecolor}{hellblauMUW}
% \usebackgroundtemplate{\includegraphics[width=\paperwidth]{MUW_hellblau.pdf}}

% \colorlet{themecolor}{skinMUW}
% \colorlet{themecolor}{blueMUW}
% \usebackgroundtemplate{\includegraphics{MUW_skin.pdf}}

%% \colorlet{themecolor}{blueMUW}
% \colorlet{themecolor}{hellblauMUW}
% \usebackgroundtemplate{\includegraphics{MUW_hellblau.pdf}}
% -----------------------------------------------
% END Set the colors
% -----------------------------------------------


% -----------------------------------------------
% START Set the width of the columns
% -----------------------------------------------
\setlength{\paperwidth}{90cm} % A0 width: 46.8in
\setlength{\paperheight}{110cm} % A0 height: 33.1in
\newlength{\sepmargin}
\newlength{\sepwid}
\newlength{\onecolwid}
\newlength{\twocolwid}
\newlength{\threecolwid}

% The following measures are used for 2 columns
\setlength{\sepmargin}{0.055\paperwidth} % Separation width (white space) between columns
\setlength{\sepwid}{0.03\paperwidth} % Separation width (white space) between columns
\setlength{\onecolwid}{0.43\paperwidth} % Width of one column
\setlength{\twocolwid}{0.9\paperwidth} % Width of two columns

% -----------------------------------------------------------
% The following measures are used for 3 columns
% \setlength{\sepmargin}{0.06\paperwidth} % Separation width (white space) between columns
% \setlength{\sepwid}{0.02\paperwidth} % Separation width (white space) between columns
% \setlength{\onecolwid}{0.28\paperwidth} % Width of one column
% \setlength{\twocolwid}{0.58\paperwidth} % Width of two columns
% \setlength{\threecolwid}{0.88\paperwidth} % Width of three columns
% \setlength{\columnsep}{30pt}

% -----------------------------------------------
% END Set the width of the columns
% -----------------------------------------------


% --------------------------------------------------------------------------------------
%	TITLE SECTION
% --------------------------------------------------------------------------------------
\setbeamertemplate{title}[left]
\setbeamertemplate{frametitle}[default][left]
% \setmainfont{Georgia}

\title{Numerical schemes for Classical\\  Chemotaxis Equations  PC-239} % Poster title

% \subtitle{IX International Congress on Industrial and Applied Mathematics \\ Valencia, Spain, 15-19th July 2019}
\author{ Daniel Acosta Soba$^1$, Alba M. Navarro Izquierdo$^2$, J. Rafael Rodr\'iguez Galv\'an$^3$} % Author(s)

\institute{Departamento de Matem\'aticas, Universidad de C\'adiz} % Institution(s)
% --------------------------------------------------------------------------------------
\pagenumbering{gobble}

\usepackage{scrpage2}
\usepackage{tikz}
\usetikzlibrary{calc}
\newcommand{\pageframe}{%
\begin{tikzpicture}[remember picture, overlay]
% page frame
\fill [themecolor] (current page.north west)
rectangle (current page.south east);
\fill [white, rounded corners=1cm] ($(current page.north west)+(1cm,-2cm)$)
rectangle ($(current page.south east)+(-1cm,2cm)$);
% \strut gives all page mark nodes the same hight.
\node[inner sep=0pt] (russell) at ($(current page.north east)+(-11cm,-11cm)$)
{\includegraphics[width=.15\textwidth]{ProjectLogo.jpg}};
\end{tikzpicture}

}
% set page style
\cehead[\pageframe]{\pageframe}
\cohead[\pageframe]{\pageframe}
\pagestyle{scrheadings}

\usepackage{hydstokes-sipdg}
\newcommand{\property}[1]{\alert{\textbf{#1}}}
%%%%%%%%%%%%%%%%%%%%%%%%%%%%%%%%%%%%%%%%%%%%%%%%%%%%%%%%%%%%%%%%%%%%%%%%%%%%%%

\begin{document}

\addtobeamertemplate{block end}{}{\vspace*{1ex}} % White space under blocks
\addtobeamertemplate{block alerted end}{}{\vspace*{0ex}} % White space under highlighted (alert) blocks
\setlength{\belowcaptionskip}{2ex} % White space under figures
\setlength\belowdisplayshortskip{1ex} % White space under equations


\begin{frame}[t] % The whole poster is enclosed in one beamer frame

\begin{columns}[t]
\begin{column}{\sepmargin}\end{column}

\begin{column}{0.97\linewidth}
    \vskip2cm
    \raggedright
    \usebeamercolor{title in headline}{\color{blueMUW}\Huge{\textbf{\inserttitle}}\\[1.5ex] \par}
    \usebeamercolor{author in headline}{\color{blueMUW}\LARGE{\insertauthor}\\[1ex]}
    \usebeamercolor{institute in headline}{\color{blueMUW}\normalsize{\insertinstitute}}{\color{blueMUW}}
    \vspace*{0.5cm}

    \rule{1.012\textwidth}{10pt}
\end{column}
\end{columns}

\vspace*{0.5cm}

\begin{columns}[t] % The whole poster consists of two major columns

\begin{column}{\sepmargin}\end{column}

\begin{column}{\onecolwid} % The first column


    \begin{block}{Introduction}
    Chemotaxis (movement of biological cells in response to
    chemical signals) was modeled by Keller-Segel in 1970.
    Although there are several models, we focus on the
    classical one, given by the following equations in
    $\Omega\subset\mathbb{R}^n$:
    \begin{subequations}\label{KSclasico}
        \begin{empheq}[left=\empheqlbrace]{align}
        &u_t= \alpha_0\Delta u - \alpha_1\nabla\cdot( u\nabla v), \quad x\in\Omega,\, t>0, \label{KSclasico:a}\\
        &v_t= \alpha_2\Delta v -\alpha_3 v+\alpha_4 u,  \quad x\in\Omega,\, t>0,\label{KSclasico:b}\\[0.2cm]
        &\grad u \cdot \nn = \grad v \cdot \nn = 0, \quad x\in\partial\Omega,\, t>0, \\[0.2cm]
        &u(x,0)=u_0(x), \enspace v(x,0)=v_0(x), \quad x\in\Omega,
        \end{empheq}
    \end{subequations}
    where $u$ and $v$ represent density of
    \alert{cells} and \alert{chemical-signal}, respectively.

        \bigskip\par\indent From an analytical point of view, a lot of
        research has been recently done (see e.g.~\cite{Winkler} and
        references therein) and interesting results about global in
        time existence, mass conservation, energy, blow-up and
        positivity of solution have been published.  However, there is
        not a large literature on \textit{numerical analysis}
        for~(\ref{KSclasico}), and reproducing former properties
        is an interesting challenge. This work is mainly
        focused on development of \textit{positivity preserving
          numerical schemes}, related to discontinous \textit{Galerkin
          methods}, which \textit{decouple} calculus of $u$ and $v$.

      \end{block}

      \begin{block}{Energy-Stable Semi-Discretization in Time}
        Given a partition of time interval $(0,T)$ into subintervals
        of size $k>0$, we approximate $u$ and $v$ at each time step
        $t^{m+1}$ by an implicit Euler scheme\footnote{However,
          this work can be applied to higher order implicit methods
          or, following \cite{anderson_high-order_2017} and references
          therein, to parallel explicit high-order approximations via
          Strong Stability Preserving (SSP) methods} as follows:
        \begin{subequations}\label{tfgschemes}
          \begin{empheq}[left=\empheqlbrace]{align}
            &\delta_t u^{m+s}-\nabla u^{m+s} +\alpha_1\nabla\cdot(u^{m+s}\nabla v^{m+s})=0,\label{TFGschemes:a}\\
            &\delta_t v^{m+s}-\alpha_2\nabla v^{m+s}+\alpha_3v^{m+s}-\alpha_4u^{m}=0, \label{TFGschemes:b}
          \end{empheq}
        \end{subequations}
        where $s\in\{0,1\}$ and $\delta_t$ is the backward difference
        operator. For this schemes, \property{energy-stability} (for
        an adequate discrete energy functional) can be shown:
      \end{block}

      \begin{block}{MPP Space Discretization for~(\ref{TFGschemes:a})}
        Let $\Th$ a mesh of $\Omega$ and let $\Uh$ be a space of
        $\mathbb{P}_k^d$ (discontinous) polynomials in elements
        $K\in\Th$ \textbf{(mejorar esta definición, como en
          \cite{anderson_high-order_2017}, sección 2.1)}. We
        fix\footnote{Results presented here might be improved to high
          order space approximations by using Bernstein polynomials,
          see\cite{anderson_high-order_2017}} $k=1$.  Let $\Vh$ be an
        space of (conforming or not) FE. For each $m\ge 0$ let
        $\vmm\in\Vh$ computed from~(\ref{TFGschemes:b}), where $\umm$
        is replaced by $P_{\Vh}(\umm)$ (its $L^2(\Omega)$--projection
        on $\Wh$)  and let $\wwmm=P_{\Uh^2}(\grad \vmm)$.

        Let us consider the following discrete problem: find
        $\umm\in\Uh$,
        \begin{equation}
        \label{discrete_problem_u}
          \int_\Omega \delta_t\umm \phi + \asip(\umm,\phi) + \agod(\umm,\phi)=0 \quad \forall\phi\in\Uh,
        \end{equation}
        where
        \begin{align*}
          \asip(u,\phi) &= \sum_{K\in\mathcal{T}_h}\int_K\nabla_h u\nabla_h \phi-\sum_{e\in\mathcal{E}_h}\int_e\left(\average{\nabla_h u\cdot\mathbf{n}_e}\jump{\phi}+\average{\nabla_h \phi\cdot\mathbf{n}_e}\jump{u}\right)\\&+\sigma\sum_{e\in\mathcal{E}_h}\int_e\frac{1}{h_e}\jump{u}\jump{\phi},
          \\
          \agod(u,\phi) &= -\sum_K \int_K u \, (\wwmm\cdot \grad\phi)
                            + \sum_{e\in\Eh}\{ u \, \wwmm \cdot \nn_e\}_\star
        \end{align*}
        and
        $$
        \{ u \, \wwmm \cdot \nn_e\}_\star=\textbf{...completar...}
        $$
        is the Godunov (upwind) flux~\cite{anderson_high-order_2017}.

      \end{block}

    \end{column}



    \begin{column}{\sepwid}  \end{column}

    \vspace*{0.5cm}

    \begin{column}{\onecolwid} %The second column
      \begin{block}{Maximum Principle Preserving (MPP)}
        Method~(\ref{discrete_problem_u}) can be recast as:
        \begin{equation}
          \label{eq:1}
          M \delta_t \umm = K \um
        \end{equation}
      \end{block}


      \vspace{0.2cm}

      \begin{block}{Test 2. Blow-up and Positivity}
      \end{block}

    \end{column}

    \begin{column}{\sepmargin} \end{column}
  \end{columns}



  \vspace*{0.5cm}
  \begin{columns}[t]
    \begin{column}{\sepmargin}\end{column}
    \begin{column}{0.97\linewidth}
      {\color{blueMUW}\rule{1.012\textwidth}{10pt}}
    \end{column}
    \begin{column}{\sepmargin}\end{column}
  \end{columns}

  \vspace*{-1cm}
  \begin{columns}[t] % Split up the two columns wide column again

    \begin{column}{\sepmargin} \end{column}
    \begin{column}{\onecolwid}
      \begin{block}{\large References}
        \vspace*{-0.5cm}
        \nocite{*} % Insert publications even if they are not cited in the poster
        {\footnotesize
          % \bibliographystyle{plainurl}
          \bibliography{bibliog.bib}}
      \end{block}
    \end{column} % End of the second column
    \begin{column}{\sepwid}  \end{column}

    \begin{column}{\onecolwid}
      \begin{block}{\large Acknowledgement}
        \vspace*{-0.5cm}
        \footnotesize Second author is partially supported by the research group FQM-315 of Junta de Andalucı\'ia and by the project: New methods for predicting, manufacturing and application of shimming (FUTURE SHIMMING) -UIC AIRBUS, and the research team: Mathematics for Computational Intelligence Systems (M$\cdot$CIS).
      \end{block}
      \vspace*{-0.5cm}

      \begin{block}{\large Contact}
        \vspace*{-0.5cm}
        \footnotesize
        $^1$\emph{daniel.acostsoba@alum.uca.es}

        $^2$\emph{alba.navarroiz@alum.uca.es}

        $^3$\emph{rafael.rodriguez@uca.es}
      \end{block}
    \end{column}

    \begin{column}{\sepmargin}\end{column} % Empty spacer column



  \end{columns} % End of all the columns in the poster

\end{frame} % End of the enclosing frame

\end{document}
